%%%%%%%%%%%%%%%%%%%%%%%%%%%%%%%%%%%%%%%%%%%%%%%%%%%%%%%%%%%%%%%%%%%%
% Diskussion und Ausblick
%%%%%%%%%%%%%%%%%%%%%%%%%%%%%%%%%%%%%%%%%%%%%%%%%%%%%%%%%%%%%%%%%%%%

\chapter{Stand der Technik}
  \label{SOTA}
Während im Grundlagenkapitel notwendige Begrifflichkeiten, Datenstrukturen, Basisalgorithmen oder Hardware-Architekturen vorgestellt werden, befasst sich dieser Abschnitt mit einer kurzen Diskussion existierender Ansätze und deren Probleme.

Je nach Themenstellung kann dieser Abschnitt auch entfallen. Eine kurze Diskussion kann in diesem Fall entweder in der Aufgabenstellung (Kapitel \ref{Grundlagen}) oder zu Beginn des eigenen Konzepts (Kapitel \ref{Konzept}) erfolgen. 
