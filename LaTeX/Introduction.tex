%%%%%%%%%%%%%%%%%%%%%%%%%%%%%%%%%%%%%%%%%%%%%%%%%%%%%%%%%%%%%%%%%%%%
% Abstract
%%%%%%%%%%%%%%%%%%%%%%%%%%%%%%%%%%%%%%%%%%%%%%%%%%%%%%%%%%%%%%%%%%%%

\chapter{Introduction}\label{Introduction}
Viewing medical images on a light box is a thing of the past. In the digital age we're living in, tablets and monitors have taken over. With this change comes the need for new software. Mulitple applications for viewing X-Ray images, MR-scans, PET-scans etc. have already been developed. However, the digital world is evolving constantly and so existing applications need to be maintained, updated and improved. In recent times, machines have been so far improved, that 4d imaging is now possible. The fourth dimension being time. Those 4d scans are called CINE scans. CINE comes from the word cinematic and stands for the timely dimension which enables a filmlike view of the scanned area.
In the last few years Artificial Intelligence (AI) has made its way in to the medical field and is becoming more and more useful. These algorithms often work different as the human brain and so the outcome of the AI's calculations is usually the only useful information for the user.
In cooporation with Marcel Früh, I have developed a simplistic App for viewing the work of an AI that has been trained to calculate the movement of a point in a 4 dimensional dataset. Doctors still use markers and printed images to make plans for treatment or discuss a possible diagnosis with colleagues. Though this method requires a physical meeting, so it would become quite impossible when the coworker is a piece of software. Selections need to be made digitally and stored in such a way. that it is easily accessible to external software like an AI programm. The main goal was to create a tool that is easy to use and does not confuse the user with to many menu options and hidden features. In the following I will give a quick overview of medical image data and their purpose, as well as the current options for viewing 4d data. I will then explain my approach when designing and planning the app. 
